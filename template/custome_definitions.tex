% --- Delimiter ---
% Set {}
% \newcommand{\set}[1]{\ensuremath{\mathopen{}\left\{#1\right\}\mathclose{}}}
\DeclarePairedDelimiter\set{\{}{\}}%
\makeatletter
\let\oldset\set
\def\set{\@ifstar{\oldset}{\oldset*}}
\makeatother
% Interpretation 
% \newcommand{\interpret}[1]{\ensuremath{\mathopen{}\left\llbracket #1 \right\rrbracket\mathclose{}}}
\DeclarePairedDelimiter\interpret{\llbracke}{\rrbracket}%
\makeatletter
\let\oldinterpret\interpret
\def\interpret{\@ifstar{\oldinterpret}{\oldinterpret*}}
\makeatother
% abs + norm value ||
\DeclarePairedDelimiter\abs{\lvert}{\rvert}%
\DeclarePairedDelimiter\norm{\lVert}{\rVert}%
\makeatletter
\let\oldabs\abs
\def\abs{\@ifstar{\oldabs}{\oldabs*}}
%
\let\oldnorm\norm
\def\norm{\@ifstar{\oldnorm}{\oldnorm*}}
\makeatother
% sytem <>
\DeclarePairedDelimiter\ezs{\langle}{\rangle}
\makeatletter
\let\oldezs\ezs
\def\ezs{\@ifstar{\oldezs}{\oldezs*}}
\makeatother
% ceiling
\DeclarePairedDelimiter{\ceil}{\lceil}{\rceil}
\makeatletter
\let\oldceil\ceil
\def\ceil{\@ifstar{\oldceil}{\oldceil*}}
\makeatother
% floor 
\DeclarePairedDelimiter{\floor}{\lfloor}{\rfloor}
\makeatletter
\let\oldfloor\floor
\def\floor{\@ifstar{\olfloor}{\oldfloor*}}
\makeatother
% parentheses () / brackets
\DeclarePairedDelimiter{\br}{(}{)}
\makeatletter
\let\oldbr\br
\def\br{\@ifstar{\oldbr}{\oldbr*}}
\makeatother
% --- Spacing ---
\newcommand{\sep}{\ensuremath{,\;}}

% --- Number sets ---
\newcommand{\R}{\ensuremath{\mathbb{R}}}
\newcommand{\N}{\ensuremath{\mathbb{N}}}
\newcommand{\Z}{\ensuremath{\mathbb{Z}}}
\newcommand{\Q}{\ensuremath{\mathbb{Q}}}
\newcommand{\CC}{\ensuremath{\mathbb{C}}}
\newcommand{\F}{\ensuremath{\mathbb{F}}}
\newcommand{\Prime}{\ensuremath{\mathbb{P}}}

% --- logic operators ---
\DeclareMathOperator{\xor}{\oplus}
\DeclareMathOperator{\impl}{\rightarrow}

% --- greek letters ---
\renewcommand{\phi}{\ensuremath{\varphi}}
\renewcommand{\theta}{\ensuremath{\vartheta}}
\NewCommandCopy{\oldpsi}{\psi}
\renewcommand{\psi}{\ensuremath{\oldpsi}}
\renewcommand{\epsilon}{\ensuremath{\varepsilon}}

% --- Operators ---
\DeclareMathOperator{\id}{id}
% for gaussion transition arrows
\newcommand{\gaussto}[1]{\ensuremath{\overset{\substack{#1}}{\leadsto}}}
\newcommand{\gaussadd}[2]{\ensuremath{\alpha_{#1}(#2)}}
\newcommand{\gaussmult}[2]{\ensuremath{\mu_{#1}(#2)}}
\newcommand{\gaussswitch}[1]{\ensuremath{\tau_{#1}}}
\newcommand{\Bild}{\ensuremath{\mathrm{Bild} \ }}
\newcommand{\Kern}{\ensuremath{\mathrm{Kern} \ }}
\DeclareMathOperator{\Real}{Re}
\DeclareMathOperator{\Image}{Im}
\newcommand{\Rg}{\ensuremath{\mathrm{Rg} \ }}
\newcommand{\Def}{\ensuremath{\mathrm{Def} \ }}
\newcommand{\abb}[1]{\ensuremath{\mathrm{Abb}(#1)}}
\newcommand*\conj[1]{\ensuremath{\overline{#1}}}
% gcd to ggT if german
% \let\gcd\relax
% \DeclareMathOperator{\gcd}{ggT}%

% --- VECTORS ---
% dynamic column-vector marco:
% > \cvec{x_1}{x_2}{x_3}....
\makeatletter
\newcommand{\cvec}[1]{%
    \begin{pmatrix}%
        #1%
        \@cvecCheckNext
}%
\newcommand{\@cvecCheckNext}{%
\@ifnextchar\bgroup{\@cvecCosumeNext}{\end{pmatrix}}%
}%
\newcommand{\@cvecCosumeNext}[1]{%
    \\ #1%
    \@cvecCheckNext
}%
\makeatother

% dynamic row-vector marco:
% > \cvec{x_1}{x_2}{x_3}....
\makeatletter
\newcommand{\rvec}[1]{%
    \begin{pmatrix}%
        #1%
        \@rvecCheckNext
}%
\newcommand{\@rvecCheckNext}{%
\@ifnextchar\bgroup{\@rvecCosumeNext}{\end{pmatrix}}%
}%
\newcommand{\@rvecCosumeNext}[1]{%
    & #1%
    \@rvecCheckNext
}%
\makeatother%
% Keept for backwards compability
\newcommand{\vecII}[3]{\ensuremath{\begin{pmatrix}[c] #1 \\ #2 \end{pmatrix}}}
\newcommand{\vecIII}[3]{\ensuremath{\begin{pmatrix}[c] #1 \\ #2 \\ #3 \end{pmatrix}}}
\newcommand{\vecIIIT}[3]{\ensuremath{(#1 \  #2 \  #3)}}
\newcommand{\matIV}[4]{\ensuremath{\begin{pmatrix}[cc] #1 & #2 \\ #3 & #4 \end{pmatrix}}}
% add custom column too matrix enviorments
\makeatletter
\renewcommand*\env@matrix[1][*\c@MaxMatrixCols c]{%
  \hskip -\arraycolsep
  \let\@ifnextchar\new@ifnextchar
  \array{#1}}
\makeatother

% --- Other ---
\newcommand{\yields}{\ensuremath{\mathrel{\Rightarrow}}}
%\renewcommand{\iff}{\xLeftrightarrow{}}
\newcommand{\todo}[1][]{{\color{red}TODO% 
        \ifx #1\empty\else: #1\fi%
}}